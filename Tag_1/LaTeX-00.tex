\documentclass[12pt,ngerman,parskip=half]{scrartcl}
%Ich bin ein Kommentar
% article, report und book sind US/englische Klassen.
% Im deutschsprachigen Raum sind KOMA Klassen besser

\author{Uwe Ziegenhagen}
\title{Mein erstes LaTeX-Dokument}

% Lokalisierung von vielen Begriffen, Silbentrennung
\usepackage{babel}
% Dummy Text
\usepackage{blindtext}
% Mikrotypografie, optischer Randausgleich
\usepackage{microtype}
% Grafiken einbinden
\usepackage{graphicx}
% Farbunterstützung laden
\usepackage{xcolor}

% Schöne Tabellen
\usepackage{booktabs}

%\usepackage{showlabels}
\usepackage{showkeys}

\usepackage{prettyref}
\newrefformat{sec}{Abschnitt~\ref{#1} auf Seite~\pageref{#1}}
\newrefformat{fig}{Grafik/Abbildung~\ref{#1}}
\newrefformat{eq}{\textup{(\ref{#1})}}
\newrefformat{tab}{Tabelle \ref{#1} auf Seite \pageref{#1}}
\newrefformat{cha}{Kapitel \ref{#1}}

\usepackage[pagewise]{lineno}
\linenumbers

% Benenne ``Abbildung'' um in ``Abb.''
\renewcaptionname{ngerman}{\figurename}{Abb.}

% Hyperlinks im Dokument, üblicherweise als letztes laden
\usepackage{hyperref}
\hypersetup{
    bookmarks=true,                     % show bookmarks bar
    unicode=false,                      % non - Latin characters in Acrobat’s bookmarks
    pdftoolbar=true,                        % show Acrobat’s toolbar
    pdfmenubar=true,                        % show Acrobat’s menu
    pdffitwindow=false,                 % window fit to page when opened
    pdfstartview={FitH},                    % fits the width of the page to the window
    pdftitle={My title},                        % title
    pdfauthor={Author},                 % author
    pdfsubject={Subject},                   % subject of the document
    pdfcreator={Creator},                   % creator of the document
    pdfproducer={Producer},             % producer of the document
    pdfkeywords={keyword1, key2, key3},   % list of keywords
    pdfnewwindow=true,                  % links in new window
    colorlinks=true,                        % false: boxed links; true: colored links
    linkcolor=blue,                          % color of internal links
    filecolor=blue,                     % color of file links
    citecolor=blue,                     % color of file links
    urlcolor=blue                        % color of external links
}

% Einfache Definition von eigenen Befehlen in LaTeX 2e Syntax
\newcommand{\person}[1]{\textsc{\textcolor{red}{#1}}}
\newcommand{\Person}[2]{\textsc{\textcolor{red}{#1}~\textcolor{gray}{#2}}}
\newcommand{\dlr}[1]{Deutsche Zentrum für Luft- und Raumfahrt}
\newcommand{\Dlr}[1]{Deutschem Zentrum für Luft- und Raumfahrt}

\begin{document}
\maketitle

\tableofcontents

%\pagebreak % \newpage

\listoffigures

\section[Einleitung und Überblick]{Einleitung in das Thema unter Berücksichtigung der Literatur im In- und Ausland}\label{sec:Einleitung}
\subsection{Literatur}

\subsubsection{Deutschland}

Hallo DLR!

Hallo DLR, ich bin ein Satz. Ich bin der zweite Satz im gleichen Absatz.

\person{Albert Einstein} hat in seiner Diss tolle Dinge bewiesen. 

\Person{Albert}{Einstein} hat in seiner Diss tolle Dinge bewiesen. 

Hallo DLR, ich bin ein Satz.  Ich bin der erste Satz im zweiten Absatz.

% PDF kann PDF, JPG und PNG
\begin{figure}
\begin{center}
\includegraphics[width=0.8\textwidth]{Bilder/Katze}
\caption{Melli 1}\label{fig:Katze}
\end{center}
\end{figure}

\blindtext[10]


\begin{figure}
\begin{center}
\includegraphics[width=0.8\textwidth]{Bilder/Katze1}
\caption[Kurzversion der caption]{Melli 2, \blindtext}
\end{center}
\end{figure}

\blindtext[10]


\begin{figure}
\begin{center}
\includegraphics[width=0.8\textwidth]{Bilder/miau}
\caption{Melli 3}
\end{center}
\end{figure}

\subsubsection{International}

sdfsdfsd


\blindtext

\section{Analyse} 

\blindtext

\blindtext[10]

\section{Fazit}

\blindtext[10]

Siehe Abbildung \ref{fig:Katze} auf dsadsf sdfsd fsd fsd sdf sdf dsf sdf sdfsd fsd sfsdfsd fsdf df  Seite~\pageref{fig:Katze}

Siehe Abschnitt \ref{sec:Einleitung}

Siehe \prettyref{sec:Einleitung}

Siehe \prettyref{fig:Katze}

\section{Tabellen}
\subsection{Eine einfache Tabelle}
%left, right, center, Absatz mit Breite
\begin{tabular}{lrcp{5cm}}
123 & 456 & 789  & Hallo, ich bin ein Text, der umgebrochen wird nach ca. 5 Zentimetern. \\
24 423 123 & 456 424234 &  32789  & Hallo, ich bin ein Text, der umgebrochen wird nach ca. 5 Zentimetern. \\
\end{tabular}

\subsection{Eine unschöne Tabelle}

\begin{tabular}{|l|r|c|p{5cm}|} \hline
Spalte 1 & Spalte 2  & Spalte 3 & Spalte 4 \\ \hline
123 & 456 & 789  & Hallo, ich bin ein Text, der umgebrochen wird nach ca. 5 Zentimetern. \\ \hline
24 423 123 & 456 424234 &  32789  & Hallo, ich bin ein Text, der umgebrochen wird nach ca. 5 Zentimetern. \\ \hline
\end{tabular} 

\blindtext

\subsection{Eine schöne Tabelle (auch mit booktabs)}

\begin{table}
\caption{Meine schöne Tabelle}\label{tab:erste}
\begin{center}
\begin{tabular}{lrcp{5cm}} \toprule[2pt]
\textbf{Spalte 1} & {\bfseries Spalte 2}  & \textbf{Spalte 3}  & \textbf{Spalte 4}  \\ \cmidrule[1pt](rl){1-4}
123 & 456 & 789  & Hallo, ich bin ein Text, der umgebrochen wird nach ca. 5 Zentimetern. \\ \midrule
24 423 123 & 456 424234 &  32789  & Hallo, ich bin ein Text, der umgebrochen wird nach ca. 5 Zentimetern. \\ \bottomrule[2pt]
\end{tabular} 
\end{center}
\end{table}

\blindtext

\end{document}
