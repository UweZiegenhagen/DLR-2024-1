\documentclass[12pt,ngerman,parskip=half]{scrartcl}
%Ich bin ein Kommentar
% article, report und book sind US/englische Klassen.
% Im deutschsprachigen Raum sind KOMA Klassen besser

\author{Uwe Ziegenhagen}
\title{Mein erstes LaTeX-Dokument}

% Lokalisierung von vielen Begriffen, Silbentrennung
\usepackage{babel}
% Dummy Text
\usepackage{blindtext}
% Mikrotypografie, optischer Randausgleich
\usepackage{microtype}
% Grafiken einbinden
\usepackage{graphicx}
% Farbunterstützung laden
\usepackage{xcolor}

% Schöne Tabellen
\usepackage{booktabs}

% Textteile auskommentieren
\usepackage{comment}

% Inhalte im Querformat
\usepackage{pdflscape}

% kompaktere Aufzählungen und Auflistungen
\usepackage{paralist}

%\usepackage{showlabels}
\usepackage{showkeys}

\usepackage{prettyref}
\newrefformat{sec}{Abschnitt~\ref{#1} auf Seite~\pageref{#1}}
\newrefformat{fig}{Grafik/Abbildung~\ref{#1}}
\newrefformat{eq}{\textup{(\ref{#1})}}
\newrefformat{tab}{Tabelle \ref{#1} auf Seite \pageref{#1}}
\newrefformat{cha}{Kapitel \ref{#1}}

\usepackage[pagewise]{lineno}
\linenumbers

% Benenne ``Abbildung'' um in ``Abb.''
\renewcaptionname{ngerman}{\figurename}{Abb.}

% Hyperlinks im Dokument, üblicherweise als letztes laden
\usepackage{hyperref}
\hypersetup{
    bookmarks=true,                     % show bookmarks bar
    unicode=false,                      % non - Latin characters in Acrobat’s bookmarks
    pdftoolbar=true,                        % show Acrobat’s toolbar
    pdfmenubar=true,                        % show Acrobat’s menu
    pdffitwindow=false,                 % window fit to page when opened
    pdfstartview={FitH},                    % fits the width of the page to the window
    pdftitle={My title},                        % title
    pdfauthor={Author},                 % author
    pdfsubject={Subject},                   % subject of the document
    pdfcreator={Creator},                   % creator of the document
    pdfproducer={Producer},             % producer of the document
    pdfkeywords={keyword1, key2, key3},   % list of keywords
    pdfnewwindow=true,                  % links in new window
    colorlinks=true,                        % false: boxed links; true: colored links
    linkcolor=blue,                          % color of internal links
    filecolor=blue,                     % color of file links
    citecolor=blue,                     % color of file links
    urlcolor=blue                        % color of external links
}

% Einfache Definition von eigenen Befehlen in LaTeX 2e Syntax
\newcommand{\person}[1]{\textsc{\textcolor{red}{#1}}}
\newcommand{\Person}[2]{\textsc{\textcolor{red}{#1}~\textcolor{gray}{#2}}}
\newcommand{\dlr}[1]{Deutsche Zentrum für Luft- und Raumfahrt}
\newcommand{\Dlr}[1]{Deutschem Zentrum für Luft- und Raumfahrt}

\begin{document}
\maketitle

\tableofcontents

%\pagebreak % \newpage

\listoffigures

\section[Einleitung und Überblick]{Einleitung in das Thema unter Berücksichtigung der Literatur im In- und Ausland}\label{sec:Einleitung}
\subsection{Literatur}

\subsubsection{Deutschland}

Hallo DLR!

Hallo DLR, ich bin ein Satz. Ich bin der zweite Satz im gleichen Absatz.

\person{Albert Einstein} hat in seiner Diss tolle Dinge bewiesen. 

\Person{Albert}{Einstein} hat in seiner Diss tolle Dinge bewiesen. 

Hallo DLR, ich bin ein Satz.  Ich bin der erste Satz im zweiten Absatz.

% PDF kann PDF, JPG und PNG
\begin{figure}
\begin{center}
\includegraphics[width=0.8\textwidth]{Bilder/Katze}
\caption{Melli 1}\label{fig:Katze}
\end{center}
\end{figure}

\blindtext[10]


\begin{figure}
\begin{center}
\includegraphics[width=0.8\textwidth]{Bilder/Katze1}
\caption[Kurzversion der caption]{Melli 2, \blindtext}
\end{center}
\end{figure}

\blindtext[10]


\begin{figure}
\begin{center}
\includegraphics[width=0.8\textwidth]{Bilder/miau}
\caption{Melli 3}
\end{center}
\end{figure}

\subsubsection{International}

sdfsdfsd


\blindtext

\section{Analyse} 

\blindtext

\blindtext[10]

\section{Fazit}

\blindtext[10]

Siehe Abbildung \ref{fig:Katze} auf dsadsf sdfsd fsd fsd sdf sdf dsf sdf sdfsd fsd sfsdfsd fsdf df  Seite~\pageref{fig:Katze}

Siehe Abschnitt \ref{sec:Einleitung}

Siehe \prettyref{sec:Einleitung}

Siehe \prettyref{fig:Katze}

\section{Tabellen}
\subsection{Eine einfache Tabelle}
%left, right, center, Absatz mit Breite
\begin{tabular}{lrcp{5cm}}
123 & 456 & 789  & Hallo, ich bin ein Text, der umgebrochen wird nach ca. 5 Zentimetern. \\
24 423 123 & 456 424234 &  32789  & Hallo, ich bin ein Text, der umgebrochen wird nach ca. 5 Zentimetern. \\
\end{tabular}

\subsection{Eine unschöne Tabelle}

\begin{tabular}{|l|r|c|p{5cm}|} \hline
Spalte 1 & Spalte 2  & Spalte 3 & Spalte 4 \\ \hline
123 & 456 & 789  & Hallo, ich bin ein Text, der umgebrochen wird nach ca. 5 Zentimetern. \\ \hline
24 423 123 & 456 424234 &  32789  & Hallo, ich bin ein Text, der umgebrochen wird nach ca. 5 Zentimetern. \\ \hline
\end{tabular} 

\blindtext

\subsection{Eine schöne Tabelle (auch mit booktabs)}

\begin{table}
\caption{Meine schöne Tabelle}\label{tab:erste}
\begin{center}
\begin{tabular}{lrcp{5cm}} \toprule[2pt]
\textbf{Spalte 1} & {\bfseries Spalte 2}  & \textbf{Spalte 3}  & \textbf{Spalte 4}  \\ \cmidrule[1pt](rl){1-4}
123 & 456 & 789  & Hallo, ich bin ein Text, der umgebrochen wird nach ca. 5 Zentimetern. \\ \midrule
24 423 123 & 456 424234 &  32789  & Hallo, ich bin ein Text, der umgebrochen wird nach ca. 5 Zentimetern. \\ \bottomrule[2pt]
\end{tabular} 
\end{center}
\end{table}

\blindtext

\begin{landscape}

\begin{table}
\caption{Meine schöne Tabelle}\label{tab:breite}
\begin{center}
\begin{tabular}{llllllllll} \toprule
Spalte 1 	&	Spalte 2	&	Spalte 3	&	Spalte 4	&	Spalte 5	&	Spalte 6	&	Spalte 7	&	Spalte 8	&	Spalte 9	\\ \midrule
0,623873745	&	0,474901105	&	0,253677024	&	0,421300135	&	0,317753079	&	0,484335636	&	0,661768957	&	0,319112573	&	0,241963422	\\
0,962663365	&	0,852142369	&	0,06898825	&	0,236980319	&	0,021900806	&	0,932129683	&	0,750964755	&	0,233371319	&	0,218181394	\\
0,986895168	&	0,121554797	&	0,75682635	&	0,675373651	&	0,435913696	&	0,785583018	&	0,113285961	&	0,99734115	&	0,834593434	\\
0,191619472	&	0,339160875	&	0,919178815	&	0,122924252	&	0,577333741	&	0,291659453	&	0,747686022	&	0,397067403	&	0,434819222	\\
0,653866036	&	0,533378911	&	0,894530474	&	0,970625118	&	0,671997475	&	0,854760893	&	0,908448324	&	0,019025629	&	0,180954297	\\
0,527026507	&	0,887612008	&	0,105895724	&	0,194269618	&	0,877885189	&	0,662238205	&	0,108907976	&	0,704585863	&	0,595493035	\\
0,150981492	&	0,820946374	&	0,029242534	&	0,087196213	&	0,965706894	&	0,873737164	&	0,927315673	&	0,724929121	&	0,623013583	\\
0,612491058	&	0,770019275	&	0,195950479	&	0,069671231	&	0,939816328	&	0,145092398	&	0,743260819	&	0,977749953	&	0,990500989	\\
0,572216109	&	0,352752129	&	0,342985261	&	0,257329935	&	0,969943788	&	0,152376351	&	0,419889675	&	0,085197078	&	0,744041536	\\
0,988795933	&	0,253652186	&	0,828999692	&	0,882658653	&	0,144497313	&	0,373646506	&	0,754580516	&	0,946613983	&	0,699131004	\\
0,888820982	&	0,356829952	&	0,539347575	&	0,757391029	&	0,866630207	&	0,397452913	&	0,621809788	&	0,763320032	&	0,26409629	\\
0,968634389	&	0,576347794	&	0,368603695	&	0,101352912	&	0,160630793	&	0,041840563	&	0,615764876	&	0,119408794	&	0,901436133	\\
0,911537139	&	0,12794752	&	0,118245281	&	0,278644721	&	0,246385374	&	0,661461589	&	0,943383049	&	0,617331103	&	0,775330492	\\
0,223298662	&	0,144625932	&	0,856649911	&	0,181574169	&	0,375180411	&	0,417656827	&	0,860773905	&	0,177830494	&	0,829291153	\\
0,733170527	&	0,608986793	&	0,153375158	&	0,329170708	&	0,190959721	&	0,470053143	&	0,994015519	&	0,189985812	&	0,994228297	\\
0,603385354	&	0,639904295	&	0,408360039	&	0,692318321	&	0,906684355	&	0,76760492	&	0,349654775	&	0,277696822	&	0,999068448	\\
0,105931955	&	0,987110256	&	0,046191488	&	0,103673011	&	0,48143021	&	0,843130487	&	0,95693351	&	0,07507434	&	0,234350793	\\
0,625017537	&	0,972991651	&	0,677371408	&	0,25863834	&	0,888636511	&	0,763234393	&	0,006687568	&	0,505613621	&	0,386779027	\\
0,363044232	&	0,653643506	&	0,105623288	&	0,578612064	&	0,744890469	&	0,250262414	&	0,699405835	&	0,711030826	&	0,877040589	\\
0,634135648	&	0,675318485	&	0,286705597	&	0,903360128	&	0,117615045	&	0,855226649	&	0,925633581	&	0,320293691	&	0,675567054	\\
0,414641296	&	0,826610536	&	0,223894364	&	0,153928769	&	0,944478128	&	0,536413139	&	0,349741995	&	0,444087033	&	0,423432418	\\ \bottomrule 
\end{tabular} 
\end{center}
\end{table}
\end{landscape}

\section{Textformatierung}

\textbf{um Text fett zu drucken}

\textsc{um Text in Kapitälchen zu drucken}

\textit{um Text kursiv zu drucken}

\textsl{um Text geneigt zu drucken}

\textit{\textbf{um Text fett kursiv zu drucken}}

\textup{um Text normal zu setzen}

\emph{um Text zu betonen}

\begin{comment} % Viel Text auskommentieren
\blindtext
\end{comment}

\begin{itemize}
	\item Hallo
	\item ich 
	\item bin 
	
	\begin{itemize}
	\item Hallo
	\item ich 
	\item bin 
	\item eine 
	\begin{itemize}
	\item Hallo
	\item ich 
	\item bin 
	\item eine 
	\item kleine 
	\item Auflistung
\end{itemize}
	\item kleine 
	\item Auflistung
\end{itemize}
	
	\item eine 
	\item kleine 
	\item Auflistung
\end{itemize}

\begin{enumerate}
	\item Hallo
	\item ich 
	\item bin 
	\item eine 
	\begin{enumerate}
	\item Hallo
	\item ich 
	\item bin 
	\item eine 
	\item kleine 
	\item Auflistung
\end{enumerate}
	\item kleine 
	\item Auflistung
\end{enumerate}

\begin{description}
\item[Apfel] ist Obst
\item[Birne] ist auch Obst
\item[Tomate] ist Gemüse
\end{description}

\subsection{paralist Beispiele}

\begin{compactitem}
	\item Hallo
	\item ich 
	\item bin 
	\item eine 
	\item kleine 
	\item Auflistung
\end{compactitem}

\begin{compactenum}
	\item Hallo
	\item ich 
	\item bin 
	\item eine 
	\item kleine 
	\item Auflistung
\end{compactenum}

\begin{compactdesc}
\item[Apfel] ist Obst
\item[Birne] ist auch Obst
\item[Tomate] ist Gemüse
\end{compactdesc}

\end{document}
