 \documentclass[12pt,ngerman,parskip=half]{scrartcl}
 
\usepackage{multirow}
\usepackage{tabularray}

\begin{document}
 
\section{Normale Tabelle}
 
 
 \begin{tabular}{|c|c|c|c|c|c|} \hline  
 a	&	b	&	c	&	d	&	e	&	f	\\ \hline
g	&	h	&	i	&	j	&	k	&	l	\\ \hline
m	&	n	&	o	&	p	&	q	&	r	\\ \hline
s	&	t	&	u	&	v	&	w	&	x	\\ \hline
\end{tabular}

\section{Multicolumn}
 
 \begin{tabular}{|c|c|c|c|c|c|} \hline
 \multicolumn{2}{|c|}{a}  &	c	&	d	&	e	&	f	\\ \hline
g	&	h	&	i	&	j	&	k	&	l	\\ \hline
m	&	n	&	o	&	p	&	q	&	r	\\ \hline
s	&	t	&	u	&	v	&	w	&	x	\\ \hline
\end{tabular}\vspace*{1em}
 
\section{Multirow}

In die obere *-Zelle kommt der \texttt{\textbackslash multirow} Befehl, die untere *-Zelle wird geleert.
 
\begin{tabular}{|c|c|c|c|c|c|} \hline
a   &  b   &	c	&	d	&	e	&	f	\\ \hline
g	&	h	&	i	&	j	&	k	&	l	\\ \hline
m	&	n	&	o	&	p	&	*	&	r	\\ \hline
s	&	t	&	u	&	v	&	*	&	x	\\ \hline
\end{tabular}
 
\begin{tabular}{|c|c|c|c|c|c|} \hline
a  &   b   &	c	&	d	&	e	&	f	\\ \hline
g	&	h	&	i	&	j	&	k	&	l	\\ \hline
m	&	n	&	o	&	p	&	\multirow[c]{2}{*}{q} 	&	r \\ \cline{1-4} \cline{6-6}
s	&	t	&	u	&	v	&		&	x	\\ \hline
\end{tabular}

\section{tabularray}

Umsetzung mittels tabularray-Paket.


\begin{tblr}{
  colspec={|c|c|c|c|c|c|},
  hline{1,2,3,4,5},
  cell{1}{1} = {c=2}{c}, % multicolumn
  cell{3}{5} = {r=2}{m}, % multirow
}
 a	&	b	&	c	&	d	&	e	&	f	\\ 
g	&	h	&	i	&	j	&	k	&	l	\\ 
m	&	n	&	o	&	p	&	q	&	r	\\ 
s	&	t	&	u	&	v	&	w	&	x	\\ 
\end{tblr}



 \end{document}
 
