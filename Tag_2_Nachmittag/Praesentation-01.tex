\documentclass[12pt,ngerman]{beamer}

\usepackage{babel}

\usetheme{moloch}

\title{Meine erste Präsentation}
\author{Uwe Ziegenhagen}
\institute{DLR}
\date{Köln, den \today}

\begin{document}

\begin{frame}

\maketitle

\end{frame}


\begin{frame}
\frametitle{Titel der Folie}
\framesubtitle{Untertitel der Folie}

\begin{itemize}
\item Hallo, ich bin
\item eine Aufzählung
\item der LaTeX
\item Beamer 
\item Klasse
\item und sehe gut aus
\end{itemize}

\end{frame}

\begin{frame}
\frametitle{Titel der Folie}
\framesubtitle{Untertitel der Folie}

\begin{equation}
- \frac{p}{2} \pm \sqrt{\left(\frac{p}{2}\right)^2 - q}
\end{equation}

\end{frame}

\begin{frame}
\frametitle{Katze}

\begin{figure}
\begin{center}
\includegraphics[width=0.8\textwidth]{Bilder/Katze}
\caption{Melli}
\end{center}
\end{figure}

\end{frame}


\begin{frame}
\frametitle{Katze}
\transdissolve % Siehe Seite 139f im Handbuch

\begin{columns}
\begin{column}{0.49\textwidth}
\begin{figure}
\includegraphics[width=0.8\textwidth]{Bilder/Katze}
\caption{Melli}
\end{figure}
\end{column}
\begin{column}{0.49\textwidth}
\begin{figure}
\includegraphics[width=0.8\textwidth]{Bilder/Katze}
\caption{Melli}
\end{figure}
\end{column}

\end{columns}

\end{frame}

\begin{frame}
\frametitle{Hallo Folie}

\begin{itemize}
\item<1-> Hallo, ich bin
\item<2-> eine Aufzählung
\item<3-> der LaTeX
\item<4-> Beamer 
\item<5-> Klasse
\item<6-> und sehe gut aus
\end{itemize}


\end{frame}



\end{document}